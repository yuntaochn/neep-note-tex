\chapter{进程管理}

操作系统的特征
并发、共享、虚拟、异步
并发是最基本的特征。

处理机管理的主要功能
进程管理、进程同步、进程通信、处理机调度

内存管理的主要功能
内存分配、内存保护、地址映射、内存扩充

设备管理的主要功能
缓冲管理、设备分配、设备处理、虚拟设备

文件管理的主要功能
文件存储空间的管理、目录管理、文件的读写管理和保护

微内核OS
足够小的内核、基于C/S模式、应用机制与策略分离原理、采用面向对象技术

前趋图
有向无循环图,记为DAG,描述进程之间执行的前后关系。

并发是OS的基本特征,进程的重要特征
动态性是进程的基本特征

PCB 进程控制块

进程状态及之间切换
以及各种事件对应的切换
就绪--> 执行 进程分配到CPU资源
执行--> 就绪 时间片用完
执行--> 阻塞 IO请求
阻塞--> 就绪 IO完成

挂起状态
处于挂起状态的进程不能接受处理机调度

引起进程创建的主要事件
用户登录、作业调度、提供服务、应用请求

引起进程撤销的主要事件
正常结束、异常结束(越界错误、保护错、非法指令、
特权指令错、运行超时、等待超时、算术运算错、IO故障)、
外界干预(操作系统敢于、父进程请求、父进程终止)

临界区

同步机制

同步准则
空闲让进、忙则等待、让权等待、有限等待

信号量 wait signal

整型信号量未完全遵守准则,不满足“让权等待”准则

互斥信号量 mutex,初值为1 
 

生产者消费者问题

锁 lock

管程
组成 名称、共享数据结构说明、过程、设置初始值
条件变量 condition 

AND信号量
信号量集

进程通信
低级工具 
高级工具 共享存储、消息传递、管道通信


引入线程
在操作系统中引入线程,是为了减少程序在并发执行时锁付出的时空开销,使OS具有更好的并发性,提高CPU的利用率。
进程是分配资源的基本单位,线程是系统调度的基本单位。

线程属性
轻型实体、独立调度和分派的基本单位、可并发执行、共享进程资源
一个进程的多个线程也可以并发执行



多线程OS中实现进程之间的同步与通信,提供的同步机制
互斥锁、读写锁、条件变量、信号

用户级线程
内核支持线程

高级调度、低级调度、中级调度
作业
作业 作业步 作业流

调度算法

作业调度与进程调度

死锁
产生原因与必要条件
原因 竞争资源和进程间推进顺序非法
必要条件 互斥条件、请求和保持、不可剥夺、循环等待

解决死锁方法
预防、避免、检测和解除
预防最容易实现
避免使资源利用率最高

银行家算法


提问方式
设计避免死锁的方法
解释说明产生死锁的原因和必要条件
NEEP矩阵

管程 是什么、组成、引入原因


\section{进程线程}
\section{处理机调度}
\section{进程同步}
PV操作是重点

互斥 \\ 
空闲让进 \\
忙则等待 \\ 
优先等待 \\ 
让权等待 \\ 

\subsection{实现临界区互斥的基本方法}
\subsubsection{软件实现方法}
掌握各种算法的思想和原理,实现互斥的四个逻辑部分,重点理解各算法在进入区、退出区做了什么
各算法的缺陷(结合四原则分析)

单标志法
定义一个turn 公用整型变量
turn的值决定了允许哪个进程可以进入临界区
同一时刻最多只允许一个进程访问临界区

turn 考虑“谦让”以便理解该概念。

违反了空闲让进的原则

双标志先检查
定义一个bool型数组flag[i],flag数组表达一元,是否希望使用临界区

违反了忙则等待的原则

双标志后检查
违反了空闲让进和优先等待的原则
会导致饥饿现象

Peterson算法
flag表达意愿,turn表达谦让

利用flag解决互斥访问
利用turn解决饥饿现象

结合了单标志法和双标志后检查法

遵循了空闲让进、忙则等待、有限等待的原则 
但是未遵循让权等待的原则

综上
检查和上锁不能一下完成,故时间片用完会导致互斥失败

\subsubsection{硬件实现方法}
利用硬件实现临界问题的方法称为低级方法,或称元方法

中断屏蔽方法
利用开关中断指令实现
开关中断是特权指令,不适合用于用户级进程
关中断
临界区
开中断

只适用于单处理机,只是用内核系统

硬件指令方法
TestAndSet指令
或称TestAndSetLock指令、TSL指令
这条指令是原子操作

Swap指令



硬件实现的缺点
不能实现让权等待
有的进程可能一直选不上,从而导致饥饿现象
优点
实现简单;适用于多处理机环境


\subsection{信号量}
很重要
\begin{itemize}
	\item 整型信号量
	\item 记录型信号量
\end{itemize}
双标志先检查法可能导致同时进入临界区
之前的所有软硬件解决方案都无法实现让权等待。
故迪杰斯特拉提出了信号量机制

整型信号量
记录型信号量

原语是由关中断、开中断指令实现的。

一对原语
wait(S) signal(S)
简称为PV操作,可以简写为P(S)和V(S)

信号量表示系统中某种资源的数量
信号量可以是一个整数,也可是复杂一些的记录。

\subsubsection{整型信号量}
整型信号量
对信号量的操作只有三种:初始化、P操作、V操作

\begin{lstlisting}	
int S = 1;
void wait(int S) {
	while (S<-0);
	S = S - 1;
}
void signal(int S) {
	S = S + 1;
}	
\end{lstlisting}

实例
\begin{lstlisting}
wait(S);
使用打印机资源
signal(S);
\end{lstlisting}

整型信号量存在的问题
不满足让权等待的原则,会发生忙等

\subsubsection{记录型信号量}
记录型信号量
整型信号量存在忙等的问题,因此提出了记录型信号量。
blcok原语
wakeup原语
记录型信号量 semaphore

\begin{lstlisting}
typedef struct {
	int value;				// 剩余资源数
	struct process *L;		// 等待队列
} semaphore;

void wait(semaphore S) {
	S.value--;
	if (S.value < 0) {		// 先减一,后判断
		block(S.L);			// 阻塞,使进程从运行态进入阻塞态,中
							// 并将其挂载到信号量S的等待队列(即阻塞队列)中
	}
}

void signal(semaphore S) {
	S.value++;
	if (S.value <= 0) {
		wakeup(S.L);		// 唤醒,唤醒等待队列中的一个进程,该进程从阻塞态变为就绪态
	}
}
\end{lstlisting}

S.value的初值表示系统中某种资源的数目。

该机制遵循了让权等待的原则,不会出现忙等现象。
因为block进行自我阻塞,放弃CPU。

注:记录型信号量是本节最重要的知识点。

\subsection{信号量实现同步(互斥)}
信号量实现进程互斥、同步、前驱关系
信号量的值 = 某种资源的剩余数量

信号量机制实现进程互斥
1. 分析并发进程的关键活动,划定临界区
2. 设置互斥信号量mutex,初值为1
3. 在进入区P(mutex)--申请资源
4. 在退出区V(mutex)--释放资源

P、V操作必须成对出现

\begin{lstlisting}
/* 信号量机制实现互斥 */
semaphore mutex = 1;	// 初始化信号量,mutex表示可以进入临界区的名额 

P1() {
	P(mutex);
	临界区
	V(mutex);
}

P2() {
	P(mutex);
	临界区
	V(mutex);
}

\end{lstlisting}

信号量实现进程同步
让本异步的进程互相配合,有序推进。

1. 分析什么地方需要实现“同步关系”,即必须保证“一前一后”执行的两个操作
2. 设置同步信号量S,初值为0
3. 在前操作之后执行V(S)
4. 在后操作之前执行P(S) 即前V后P

\begin{lstlisting}
semaphore S = 0;

P1() {
	代码1;
	代码2;
	V(S);
	代码3;
}

P2() {
	P(S);
	代码4;
	代码5;
	代码6;
}

// 这就保证了代码2一定是在代码4之前执行的
\end{lstlisting}


信号量实现前驱关系
1. 要为每一对前驱关系各设置一个同步信号量
2. 在前操作之后对相应的同步信号量执行V操作
3. 在后操作之前对相应的同步信号量执行P操作

故
互斥信号量初值为1
同步信号量初值为0

\subsection{经典问题}
经典问题

经典同步互斥问题
\begin{itemize}
	\item 生产者-消费者问题
	\item 多生产者-多消费者问题
	\item 吸烟者问题
	\item 读者-写者问题
	\item 哲学家就餐问题
\end{itemize}

\paragraph{生产者-消费者问题}
PV操作题目分析步骤
1. 关系分析,找出题目中描述的各个进程,分析它们之间的同步、互斥关系
2. 整理思路,根据各进程的操作流程确定P、V操作的大致顺序
3. 设置信号量。设置需要的信号量,并根据题目条件确定信号量初值。
互斥信号量一般为1,同步信号量的初始值要看对应资源的初始值是多少。


\begin{lstlisting}
semaphore mutex = 1;	// 互斥信号量,实现对缓冲区的互斥访问
semaphore empty = n;	// 同步信号量,表示空闲缓冲区的数量
semaphore full = 0;		// 同步信号量,表示产品的数量,亦即非空缓冲区的数量
// 缓冲区的数量,产品的数量
producer() {
	while(1) {
		生产一个产品;
		P(empty);
		
		P(mutex);
		把产品放入缓冲区
		V(mutex);
		
		V(full);
	}
}

consumer() {
	while(1) {
		P(full);
		
		P(mutex);
		从缓冲区取出一个产品
		V(mutex);
		
		V(empty);
		使用产品
	}
}

\end{lstlisting}
互斥信号量在不同的进程中P、V
同步信号量在同一进程中P、V

思考,能否改变相邻PV操作的顺序?
不能,会导致死锁

因此,实现互斥的P操作,一定要在实现同步的P操作之后。

V操作不会导致进程阻塞,因此两个V操作可以互换。

生产者消费者问题是一个互斥、同步的综合问题。
要发现问题中隐含的两对同步关系。

\paragraph{多生产者-多消费者问题}
父、母、儿、女 四者
父亲 -苹果- 盘子 -- 女儿
母亲 -橘子- 盘子 -- 儿子
 
1. 分析关系。找出题目中描述的各个进程,分析它们之间的同步、互斥关系。
2. 整理思路。根据各进程的操作流程确定P、V操作的大致顺序

互斥关系
对缓冲区(盘子)的访问要互斥进行

同步关系(一前一后)
1. 父亲将苹果放入盘子后,女儿才能取苹果
2. 母亲将橘子放入盘子后,儿子才能取橘子
3. 只有盘子为空时,父亲或母亲才能放入水果

互斥 在临界区前后分别PV
同步 前V后P

\begin{lstlisting}
semaphore mutex = 1;
semaphore apple = 0;
semaphore orange = 0;
semaphore plate = 1;

dad() {
	while(1) {
		准备一个苹果
		P(plate);
		
		P(mutex);
		把苹果放入盘子;
		V(mutex);
		
		V(apple);
	}
}

mom() {
	while(1) {
	
		P(plate);
		
		P(mutex);
		把橘子放入盘子;
		V(mutex);
		
		V(orange);
	}
}

daughter() {
	while(1) {
		P(apple);

		p(mutex);
		从盘子中取出苹果;
		V(mutex);
		
		V(plate);
		吃掉苹果;
	}
}

son() {
	while(1) {
		P(orange);
		
		P(mutex);
		从盘子中取出橘子;
		V(mutex);
		
		V(plate);
		吃掉橘子;
	}
}

\end{lstlisting}
对于此问题,即使不设置mutex也是可以的,并不会导致死锁等问题。
因为apple、orange、plate三个同步信号量最多只有一个是1。
故如果缓冲区大小为1时,有可能不需要设置mutex。

要把“一前一后”发生的事情看作两种事件的前后关系。

从事件的角度分析
把进程行为的前后关系抽象为一对事件的前后关系

\paragraph{吸烟者问题}
进程同步互斥问题

1. 关系分析
互斥
桌子 抽象为容量为1的缓冲区,互斥访问
组合1 
组合2 
组合3

同步关系(从事件的角度分析)
桌上有组合1 --> 第一个抽烟者取走东西
桌上有组合2 --> 第二个抽烟者取走东西
桌上有组合3 --> 第三个抽烟者取走东西
发出完成信号 --> 供应者将下一个组合放到桌上
前V后P

2. 整理思路

\begin{lstlisting}
semaphore offer1 = 0;
semaphore offer2 = 0;
semaphore offer3 = 0;
semaphore finish = 0;
int i = 0;		
provider() {
	while(1) {
		
	
		P(finish);
	}
}

smoker1() {
	while(1) {
		P(offer1);
		从桌上拿走组合一,卷烟,抽掉;
		V(finish);
	}
}

smoker2() {
	while(1) {

	}	
}

smoker3() {
	while(1) {

	}
}

\end{lstlisting}

\paragraph{读者-写者问题}
条件
多个读者可同时读
只允许一个写者写信息
写操作完成之前不允许其他读者写者工作
写者写操作之前,已有读者和写者全部退出

分析关系
写进程、都进程
互斥关系
写进程-写进程
写进程-读进程
读进程之间不存在互斥问题

关键  
定义count变量,并互斥访问之

\begin{lstlisting}
int count = 0;		// 记录当前的读者数量
smeaphore mutex = 1;	// 保护更新count变量时的互斥
semaphore rw = 1;	// 互斥访问共享文件
int w = 1;			// 保证公平,以避免读者优先
writer() {
	while(1) {
		P(w);		//  在无写进程时进入,保证公平 
		P(rw);		// 加锁 
		写文件;
		V(rw);		// 解锁
		V(w);	
	}
}

reader() {
	while(1) {
		P(w);			// 在无写进程时进入 
		P(mutex);		// 互斥访问count变量 
		if(count == 0) 
			P(rw);
		count ++;		
		V(mutex);		// 释放互斥变量mutex
		V(w);			// 恢复对共享文件的访问 	
		
		读文件;
		
		P(mutex);		
		count --;
		if(count == 0)
			V(rw);
		V(mutex);
	}
}
\end{lstlisting}

读读不互斥
写-其他进程互斥
设置计数器count 加锁 解锁


\paragraph{哲学家进餐问题}
关系分析
对筷子的访问时互斥关系

整理思路

如何避免临界资源的分配不当造成的死锁现象,是其精髓所在。

3. 信号量设置
定义信号量数组 chopsticks[5]={1, 1, 1, 1, 1},实现对筷子的互斥访问。
对哲学家按0-4编号,哲学家坐标的筷子编号为i,右边的筷子编号为(i+5)\%5。


如何避免死锁
1. 可以添加限制条件,比如最多允许四个哲学家同时进餐,如此可保证至少一人可以拿到两只筷子
	设置初值为4的互斥信号量
2. 要求奇数号的哲学家必须先拿左边的筷子,偶数号的哲学家相反。
3. 仅当一个哲学家左右两只筷子都可以使用时,才可以拿起筷子
\begin{lstlisting}
semaphore chopstick[5] = {1, 1, 1, 1, 1};
semaphore mutex = 1;
Pi() {
	while(1) {
		P(mutex);
		P(chopstick[i]);
		P(chopstick[(i+1)%5]);
		V(mutex);				// 两只筷子都拿起来了
		
		吃饭
		
		V(chopstick[i]);
		V(chopstick[(i+1)%5]);
		思考
	}
}
\end{lstlisting}

解决进餐问题的关键在于解决死锁。
每个进程都需要同时持有两个临界资源,故有死锁的隐患。

\subsection{管程}
% 信号量机制存在问题,编写程序困难,易出错。
管程 —— 一种高级同步机制。
管程是由一组数据及定义在这组数据之上的对这组数据的操作组成的软件模块。

组成
1. 局部于管程的共享数据结构说明
2. 对该数据结构进行操作的一组过程;(亦即函数)
3. 对局部于管程的共享数据设置初始值的语句
4. 管程有一个名字

基本特征
1. 
2. 各外部进程/线程只能通过管程提供的特定“入口”才能访问共享数据
3. 每次只允许一个进程在管程内执行某个内部过程



管程解决生产者-消费者问题
\begin{lstlisting}
monitor ProducerConsumer
	condition full, empty;
	int count = 0;
	void insert(Item item) {
		
		
	}
	Item remove() {
	
		return remove_item();
	}

end monitor

producer() {
	while(1) {
		item = 生产一个产品;
		ProducerConsumer.insert(item);
	}
}

consumer() {
	while(1) {
		item = ProeucerConsumer.remove();
		消费产品item;
	}
}
\end{lstlisting}

引入管程的目的,更方便的实现进程互斥和同步。


\section{死锁}

\subsection{死锁概念}
\subsection{死锁产生原因}
系统资源竞争
进程推进顺序非法
信号量使用量不当

死锁产生的必要条件
产生死锁必须同时满足四个条件,只要其中任一条件不成立,就不会发生死锁。

\begin{itemize}
	\item 互斥
	\item 不剥夺
	\item 请求和保持
	\item 循环等待 此为必要不充分条件
			发生死锁是必有循环等待,但发生循环等待未必死锁
\end{itemize}

死锁 
饥饿
死循环

\subsection{死锁的处理策略}

预防死锁 破坏四个必要条件之一
避免死锁 防止系统进入不安全状态 如银行家算法

预防死锁 静态策略
避免死锁 动态策略
死锁的检测和接触

预防
破坏互斥条件
把互斥使用的资源改造为允许共享使用,比如SPOOLing技术



疑问 
进程管理信号量设置
何时设置empty与full
何时只需设置empty



