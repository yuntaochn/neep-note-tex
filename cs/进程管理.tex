\chapter{进程管理}

操作系统的特征
并发、共享、虚拟、异步
并发是最基本的特征。

处理机管理的主要功能
进程管理、进程同步、进程通信、处理机调度

内存管理的主要功能
内存分配、内存保护、地址映射、内存扩充

设备管理的主要功能
缓冲管理、设备分配、设备处理、虚拟设备

文件管理的主要功能
文件存储空间的管理、目录管理、文件的读写管理和保护

微内核OS
足够小的内核、基于C/S模式、应用机制与策略分离原理、采用面向对象技术

前趋图
有向无循环图,记为DAG,描述进程之间执行的前后关系。

并发是OS的基本特征,进程的重要特征
动态性是进程的基本特征

PCB 进程控制块

进程状态及之间切换
以及各种事件对应的切换
就绪--> 执行 进程分配到CPU资源
执行--> 就绪 时间片用完
执行--> 阻塞 IO请求
阻塞--> 就绪 IO完成

挂起状态
处于挂起状态的进程不能接受处理机调度

引起进程创建的主要事件
用户登录、作业调度、提供服务、应用请求

引起进程撤销的主要事件
正常结束、异常结束(越界错误、保护错、非法指令、
特权指令错、运行超时、等待超时、算术运算错、IO故障)、
外界干预(操作系统敢于、父进程请求、父进程终止)

临界区

同步机制

同步准则
空闲让进、忙则等待、让权等待、有限等待

信号量 wait signal

整型信号量未完全遵守准则,不满足“让权等待”准则

互斥信号量 mutex,初值为1 
 

生产者消费者问题

锁 lock

管程
组成 名称、共享数据结构说明、过程、设置初始值
条件变量 condition 

AND信号量
信号量集

进程通信
低级工具 
高级工具 共享存储、消息传递、管道通信


引入线程
在操作系统中引入线程,是为了减少程序在并发执行时锁付出的时空开销,使OS具有更好的并发性,提高CPU的利用率。
进程是分配资源的基本单位,线程是系统调度的基本单位。

线程属性
轻型实体、独立调度和分派的基本单位、可并发执行、共享进程资源
一个进程的多个线程也可以并发执行



多线程OS中实现进程之间的同步与通信,提供的同步机制
互斥锁、读写锁、条件变量、信号

用户级线程
内核支持线程

高级调度、低级调度、中级调度
作业
作业 作业步 作业流

调度算法

作业调度与进程调度

死锁
产生原因与必要条件
原因 竞争资源和进程间推进顺序非法
必要条件 互斥条件、请求和保持、不可剥夺、循环等待

解决死锁方法
预防、避免、检测和解除
预防最容易实现
避免使资源利用率最高

银行家算法




