\chapter{文件管理}
关键词
重点时文件系统的结构及其实现,即文件系统的逻辑结构和物理结构。
以及磁盘相关知识点。

关键词
文件控制块、物理分配方法、索引结构


\section{文件系统}
\subsection{文件逻辑结构}
流式文件
记录式文件 顺序文件 索引文件 索引顺序文件

\subsection{目录结构}
文件控制块FCB
树形目录结构


\subsection{文件共享}
基于索引结点(硬链接)
基于符号链(软链接)


\subsection{文件保护}


\section{文件系统的实现}
\subsection{目录实现}
\subsection{文件分配}
\subsection{文件存储空间管理}

求FAT的最大长度,支持的文件最大长度。
习题6 

\section{磁盘管理}
\subsection{访问时间}
\subsection{磁盘调度算法}
\subsection{磁盘管理}


文件系统的结构及其实现,磁盘相关

文件系统
文件控制块、物理分配方法、索引结构、
磁盘
磁盘特性和结构、磁盘调度算法、磁盘相关性能

文件的逻辑结构
目录结构的实现
系统调用
文件的物理结构
存储空间的管理
文件共享保护

\section{文件系统}

\subsection{文件的逻辑结构}
文件的逻辑结构
	无结构文件(流式文件) 
	有结构文件(记录式文件) 
		顺序文件
			顺序存储或者链式存储
			串结构 顺序结构 
		索引文件
		索引顺序文件

\subsection{目录结构}
FCB的有序集合称为文件目录
一个FCB就是一个文件目录项

索引结点
磁盘索引结点、内存索引结点

引入无环图目录的目的是实现文件共享
无环图 共享计数器 计数器为0时才真正删除该结点


索引结点(FCB的改进)
FCB中暂时不用的信息放入索引结点中

索引结点机制,提升了文件检索速度


% 重点内容 
\subsection{文件物理结构}
即文件分配方式

文件的物理结构/文件分配方式
文件存储空间管理

索引分配
m级访问,m+1次读磁盘(顶级索引表未调入内存的情况下)

链接分配
索引分配

混合索引分配
Unix采用13个地址,0-9 直接地址,10-一级索引,11-二级索引,12-三级索引 

文件目录与目录文件

计算逻辑地址到物理地址的转换,以及查找次数。

文件共享
基于符号链的文件共享
软链接、硬链接


\subsection{文件存储空间管理}

空闲表法
空闲链表法
位示图法 字号,位号  常考 
成组链接法 UNIX超级块 


\subsection{磁盘管理}

磁盘调度算法
SCAN
CSCAN
LOOK
CLOOK






2020年4月14日凌晨观看王道2020视频教程
