
万能公式,令 $u = tan\frac{x}{x}$  

$ sinx = \frac{2u}{1+u^2}$ 

$ cosx = \frac{1-u^2}{1+u^2}$ \\

泰勒公式展开 

$ {e}^{x}=1+x+\frac{1}{2}{x}^{2}+\frac{1}{6}{x}^{3}+ \dots + \frac{1}{n!}{x}^{n}+o\left({x}^{n}\right)$

$\mathrm{ln}\left(1+x\right)=x-\frac{1}{2}{x}^{2}+\frac{1}{3}{x}^{3}+ \dots + {\left(-1\right)}^{n}{x}^{n}+o\left({x}^{n}\right)$

$\frac{1}{1-x}=1+x+{x}^{2}+\dots +{x}^{n}+o\left({x}^{n}\right)$ \\

基本求导公式

${\left(\mathrm{arcsin}x\right)}^{\prime }=\frac{1}{\sqrt{1-{x}^{2}}}$

${\left(\mathrm{arccos}x\right)}^{\prime }=-\frac{1}{\sqrt{1-{x}^{2}}}$

${\left(\mathrm{arctan}x\right)}^{\prime }=\frac{1}{1+{x}^{2}}$

${\left(\mathrm{arccot}X\right)}^{\prime }=-\frac{1}{1+{x}^{2}}$

${\left(\mathrm{tan}x\right)}^{\prime }={\mathrm{sec}}^{2}x$

${\left(\mathrm{cot}x\right)}^{\prime }=-{\mathrm{csc}}^{2}x$

${\left(\mathrm{sec}x\right)}^{\prime }=\mathrm{sec}x\cdot \mathrm{tan}x$

${\left(\mathrm{csc}x\right)}^{\prime }=-\mathrm{csc}x\cdot \mathrm{cot}x$ \\

曲率 

$k=\frac{\left|{y}^{\prime \prime }\right|}{{\left(1+{{y}^{\prime }}^{2}\right)}^{\frac{3}{2}}}$ \\

基本积分公式 

$\int \frac{1}{x}dx=\mathrm{ln}\left|x\right|+C$

$\int \frac{1}{1+{x}^{2}}dx=\mathrm{arctan}x+C$

$\int \frac{1}{\sqrt{1-{x}^{2}}}dx=\mathrm{arcsin}x+{c}_{1}=-\mathrm{arccos}x+{c}_{2}$

$\int \mathrm{tan}xdx=-\mathrm{ln}\left|\mathrm{cos}x\right|+c$

$\int \mathrm{cot}x=\mathrm{ln}\left|\mathrm{sin}x\right|+C$

$\int \mathrm{csc}xdx=\int \frac{1}{\mathrm{sin}x}dx=\frac{1}{2}\mathrm{ln}\left|\frac{1-\mathrm{cos}x}{1+\mathrm{cos}x}\right|+C = \mathrm{ln}\left|\mathrm{tan}\frac{x}{2}\right|+c=\mathrm{ln}\left|\mathrm{csc}x-\mathrm{cot}x\right|+c$

$\int \mathrm{sec}xdx=\int \frac{1}{\mathrm{cos}x}dx=\frac{1}{2}\mathrm{ln}\left|\frac{1+\mathrm{sin}x}{1-\mathrm{sin}x}\right|+c=\mathrm{ln}\left|\mathrm{sec}x+\mathrm{tan}x\right|+c$

$\int {\mathrm{sec}}^{2}xdx=\mathrm{tan}x+c$

$\int {\mathrm{csc}}^{2}xdx=-\mathrm{cot}x+C$

$\int \mathrm{sec}x\cdot \mathrm{tan}xdx=\mathrm{sec}x+c$$\int \mathrm{csc}x\cdot \mathrm{cot}xdx=-\mathrm{csc}x+C$

$\int \frac{1}{{a}^{2}+{x}^{2}}dx=\frac{1}{a}\mathrm{tan}\frac{x}{a}+c$

$\int \frac{1}{{a}^{2}-{x}^{2}}dx=\frac{1}{2a}\mathrm{ln}\left|\frac{a+x}{a-x}\right|+c$

$\int \frac{1}{\sqrt{{a}^{2}-{x}^{2}}}dx=\mathrm{arcsin}\frac{x}{a}+c$

$\int \frac{1}{\sqrt{{x}^{2}±{a}^{2}}}dx=\mathrm{ln}\left|{x}^{2}±{a}^{2}\right|+c$

${I}_{n}={\int }_{0}^{\frac{\pi }{2}}{\mathrm{sin}}^{n}xdx={\int }_{0}^{\frac{\pi }{2}}{\mathrm{cos}}^{n}\lambda dx$

${I}_{n}=\frac{n-1}{n}{I}_{n-2}$

当n为奇数时,

${I}_{n}=\frac{n-1}{n}\cdot \frac{n-3}{n-2}\cdots \frac{4}{5}\cdot \frac{2}{3}$ 

$I_1 = 1$

当n为偶数时,

${I}_{n}=\frac{n-1}{n}\cdot \frac{n-3}{n-2}\cdots \frac{3}{4}\cdot \frac{1}{2}\cdot \frac{\pi }{2}$ 

$I_0 = \frac{\pi}{2}$



























