\chapter{函数极限连续}
极限

十大公式 \\
$e^x = $ \\ 
$sinx = $ \\
$cosx = $ \\ 
$ln(1+x) = $ \\
$\frac{1}{1-x} = $ \\
$\frac{1}{1+x} = $ \\
${(1+x)}^a = $ \\
$tanx = $ \\
$arcsinx = $ \\
$arctanx = $ \\

间断点
前提: $f(x)$在左右两侧均有定义
$$ lim_{x0+} f(x)  lim_{x0-}f(x) f(x0)  $$
第一类间断点 1,2皆存在,1,2不等,为跳跃间断点
1,2相等,但不等于3,为可去间断点

第二类间断点
1,2至少一个不存在,且等于无穷,无穷间断点
1,2至少一个不存在,且等于振荡不存在,振荡间断点

成为间断点的必要条件是两侧均有定义


先斩后奏
单调有界
夹逼准则
