\chapter{多元微分}


多元微分的题型

\section{概念}
各种术语的概念,及公式。

\noindent
偏导数连续 $\Longrightarrow$ 可微 $\Longrightarrow$ 偏导数存在(某方向双侧) \\
偏导数连续 $\Longrightarrow$ 可微 $\Longrightarrow$ 连续 $\Longrightarrow$ 极限存在(全方向) \\
偏导数存在 $\Longrightarrow$ 可微 $\Longrightarrow$ 方向导数存在(某方向单侧) \\

注
不能用洛必达法则、单调有界准则 \\
$dz = A\Delta x + B\Delta y + o(\rho)$ \\
$dz = A\Delta x + B\Delta y + o(\sqrt{x^2+y^2})$ \\


驻点

\section{复合函数求导}
求导方法
1. 链式求导法则 \\ 
2. 全导数 \\
3. 全微分形式不变性 \\



\section{隐函数求导}
一个方程的情形

方程组的情形
雅克比矩阵,公式法 

方程组简单情形



\section{多元函数的极值、最值}
题型包括条件极值和无条件极值

\subsection{无条件极值}
求解步骤
1. 令偏导数=0, 求得(x,y)
2. 求二阶偏导数
$$
A = f_{11}^{''} \quad{}  B = f_{12}^{''} \quad{} C = f_{22}^{''}
$$
其中若$B^2 - AC < 0$,极值存在,
若$A > 0$,则为极小值,若$A < 0$,则为极大值,
若$A = 0$,则失效,另外计算。


\subsection{条件极值}
运用拉格朗日乘数法,构造
$F(x, y, z, \lambda, \mu) = 0$ \\
分别求偏导数,令其=0, 求得$(x, y, z)$。
优先求$(x, y, z)$。 

\section{偏微分方程}

$z(x,y) = \int f_x^{'}(x,y) dx + \varphi(y)$ \\
代入已知条件求得$\varphi(y)$ \\
求得$z(x,y)$ \\
