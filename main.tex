%!TEX program = xelatex
\documentclass{book}
\usepackage[a4paper,left=3cm,right=3cm]{geometry}
\usepackage{ctex}
\usepackage{fancyhdr}
\pagestyle{fancy}
\renewcommand{\headrulewidth}{0pt} %改为0pt即可去掉页眉下面的横线
\lfoot{} %这条语句可以让页码出现在下方
\usepackage{float}
\begin{titlepage}
	\title{考研笔记}
	\author{张云涛}
\end{titlepage}

\begin{document}
\maketitle

\part{数学}
\chapter{函数极限连续}
极限

十大公式 \\
$e^x = $ \\ 
$sinx = $ \\
$cosx = $ \\ 
$ln(1+x) = $ \\
$\frac{1}{1-x} = $ \\
$\frac{1}{1+x} = $ \\
${(1+x)}^a = $ \\
$tanx = $ \\
$arcsinx = $ \\
$arctanx = $ \\

间断点
前提: $f(x)$在左右两侧均有定义
$$ lim_{x0+} f(x)  lim_{x0-}f(x) f(x0)  $$
第一类间断点 1,2皆存在,1,2不等,为跳跃间断点
1,2相等,但不等于3,为可去间断点

第二类间断点
1,2至少一个不存在,且等于无穷,无穷间断点
1,2至少一个不存在,且等于振荡不存在,振荡间断点

成为间断点的必要条件是两侧均有定义


先斩后奏
单调有界
夹逼准则

\chapter{一元微分}

曲线凹凸性的判断
拐点

驻点

曲率

间断点类型

特殊函数的导数

几何应用
证明应用
物理应用

中值定理

洛尔定理
拉格朗日中值定理
柯西中值定理

关于导数
确定是否为变限积分
比如$sin(x-t)$积分项中含有此式,须设$u=x-t$,或许可化为变限积分,
求导时须注意。





\chapter*{一元积分}

\chapter{多元微分}


多元微分的题型

\section{概念}
各种术语的概念,及公式。

\noindent
偏导数连续 $\Longrightarrow$ 可微 $\Longrightarrow$ 偏导数存在(某方向双侧) \\
偏导数连续 $\Longrightarrow$ 可微 $\Longrightarrow$ 连续 $\Longrightarrow$ 极限存在(全方向) \\
偏导数存在 $\Longrightarrow$ 可微 $\Longrightarrow$ 方向导数存在(某方向单侧) \\

注
不能用洛必达法则、单调有界准则 \\
$dz = A\Delta x + B\Delta y + o(\rho)$ \\
$dz = A\Delta x + B\Delta y + o(\sqrt{x^2+y^2})$ \\


驻点

\section{复合函数求导}
求导方法
1. 链式求导法则 \\ 
2. 全导数 \\
3. 全微分形式不变性 \\



\section{隐函数求导}
一个方程的情形

方程组的情形
雅克比矩阵,公式法 

方程组简单情形



\section{多元函数的极值、最值}
题型包括条件极值和无条件极值

\subsection{无条件极值}
求解步骤
1. 令偏导数=0, 求得(x,y)
2. 求二阶偏导数
$$
A = f_{11}^{''} \quad{}  B = f_{12}^{''} \quad{} C = f_{22}^{''}
$$
其中若$B^2 - AC < 0$,极值存在,
若$A > 0$,则为极小值,若$A < 0$,则为极大值,
若$A = 0$,则失效,另外计算。


\subsection{条件极值}
运用拉格朗日乘数法,构造
$F(x, y, z, \lambda, \mu) = 0$ \\
分别求偏导数,令其=0, 求得$(x, y, z)$。
优先求$(x, y, z)$。 

\section{偏微分方程}

$z(x,y) = \int f_x^{'}(x,y) dx + \varphi(y)$ \\
代入已知条件求得$\varphi(y)$ \\
求得$z(x,y)$ \\

\chapter*{多元积分}

\chapter*{空间几何}

\chapter{微分方程}

解题方法
首先确定方程类型,然后选择解题方法

方程形式

关键
$${(uv)}^{'} = u^{'}v + uv^{'}$$

曲率

\chapter*{无穷级数}

常数项级数
幂级数
福利叶级数

收敛区间 开区间
收敛半径 R
收敛域   考虑端点



\include{math/矩阵}
\chapter*{线性方程组}

\chapter*{特征值与特征向量}

\chapter*{二次型}

\chapter{随机变量及其分布}
本章主要两部分内容
分布函数和正态分布

\begin{itemize}
	\item 分布函数
	\item 正态分布
\end{itemize}



\begin{table}
	\centering
	\caption{变量分布}
	\begin{tabular}{|l|l|l|l|}
		\hline
		分布函数 &  &  &  \\ \hline
		0-1分布 &  &  &  \\ \hline
		二项分布 &  &  &  \\ \hline
		泊松分布 &  &  &  \\ \hline
		几何分布 &  &  &  \\ \hline
		超几何分布 &  &  &  \\ \hline
	\end{tabular}
\end{table}

重点考虑泊松分布和几何分布

\chapter*{多维随机变量及其分布}

事件的关系与运算

几何概型

条件概率、乘法公式、独立性

\include{math/数字特征}
\include{math/大数定律中心极限定理}
\chapter*{数理统计基本概念}

\chapter*{参数估计与假设检验}



\part{计算机}
\include{cs/线性表}
\chapter*{栈和队列}

\chapter*{树、二叉树}

\chapter{图}

关键词 \\
存储机构之邻接矩阵、邻接表 \\
遍历算法之深度优先、广度优先 \\ 
生成树之prim算法、kruscale算法 \\
最短路径之dijkstra算法、fluid算法 \\
拓扑排序之AOV网 \\
关键路径之AOE网 \\ 

\section{基本概念}
连通子图为连通分量 \\

\noindent 
对于连通图 \\
无向连通图,n个顶点,最少有n-1条边。 \\
有向连通图,n个顶点,最少有n条边,循环即可。 \\

\section{存储结构}

\section{遍历算法}
深度优先 广度优先 \\
深栈 

\section{最小生成树}

\section{最短路径}

\section{拓扑排序}

AOV网
是一种以顶点表示活动、以边表示活动的先后次序,并且没有回路的有向图。

\section{关键路径}


\chapter*{查找}

最短路径

关键路径

查找判定树
\begin{figure}[h]
	\centering
	\includegraphics[scale=0.4]{images/cs/KMPnextval.jpg}
	\caption[kmp]{kmp-nextval}
\end{figure}

\chapter{排序}

% 此处插入一个表格,统计各种排序算法
% Please add the following required packages to your document preamble:
% \usepackage{multirow}
% \usepackage{graphicx}
\begin{table}[H]
	\centering
	\resizebox{\textwidth}{!}{%
		\begin{tabular}{|c|l|l|l|l|l|}
			\hline
			&        & 稳定性 & 时间复杂度 & 空间复杂度 & 其他 \\ \hline
			\multirow{3}{*}{插入排序} & 直接插入排序 &     &       &       &    \\ \cline{2-6} 
			& 折半插入排序 &     &       &       &    \\ \cline{2-6} 
			& 希尔排序   &     &       &       &    \\ \hline
			\multirow{2}{*}{交换排序} & 冒泡排序   &     &       &       &    \\ \cline{2-6} 
			& 快速排序   &     &       &       &    \\ \hline
			\multirow{2}{*}{选择排序} & 简单选择排序 &     &       &       &    \\ \cline{2-6} 
			& 堆排序    &     &       &       &    \\ \hline
			& 归并排序   &     &       &       &    \\ \hline
			& 基数排序   &     &       &       &    \\ \hline
		\end{tabular}%
	}
\end{table}

\section{插入排序}
直接插入排序
折半插入排序
希尔排序

\section{交换排序}
冒泡排序
快速排序

\section{选择排序}
简单选择排序
堆排序

\section{归并排序}
\section{基数排序}

非比较排序
计数排序、桶排序、基数排序

\section{排序小结}
\paragraph{时间复杂度}
快些以$nlog_2 n$的速度归队 \\
$nlog_2 n$ 快速排序 希尔排序 归并排序 堆排序 \\
其他都是$O(n^2)$,基数排序除外。 \\

\paragraph{空间复杂度}
快速排序$O(log_2 n)$,归并排序 $O(n)$,基数排序$O(r_d)$,其余都是$O(1)$。

\paragraph{原始序列有序}
直接插容易插,起泡起的好,即初始序列已经有序,时间复杂度则为$O(n)$。

\paragraph{不稳定排序算法}
情绪不稳定,快些选堆好友。 \\
不稳定--快速排序、希尔排序、简单选择排序、堆排序 \\
其他都是稳定的。 \\

\paragraph{最终位置}
一趟能否使某个关键字到达最终位置 \\
交换类和选择类的都可以到达最终位置 \\

\paragraph{比较次数}
关键字比较次数与原始序列无关 \\
简单选择排序和折半插入排序 \\

\paragraph{排序趟数}
排序算法的排序趟数和原始序列有关——交换类的排序。 \\



\chapter{进程管理}

操作系统的特征
并发、共享、虚拟、异步
并发是最基本的特征。

处理机管理的主要功能
进程管理、进程同步、进程通信、处理机调度

内存管理的主要功能
内存分配、内存保护、地址映射、内存扩充

设备管理的主要功能
缓冲管理、设备分配、设备处理、虚拟设备

文件管理的主要功能
文件存储空间的管理、目录管理、文件的读写管理和保护

微内核OS
足够小的内核、基于C/S模式、应用机制与策略分离原理、采用面向对象技术

前趋图
有向无循环图,记为DAG,描述进程之间执行的前后关系。

并发是OS的基本特征,进程的重要特征
动态性是进程的基本特征

PCB 进程控制块

进程状态及之间切换
以及各种事件对应的切换
就绪--> 执行 进程分配到CPU资源
执行--> 就绪 时间片用完
执行--> 阻塞 IO请求
阻塞--> 就绪 IO完成

挂起状态
处于挂起状态的进程不能接受处理机调度

引起进程创建的主要事件
用户登录、作业调度、提供服务、应用请求

引起进程撤销的主要事件
正常结束、异常结束(越界错误、保护错、非法指令、
特权指令错、运行超时、等待超时、算术运算错、IO故障)、
外界干预(操作系统敢于、父进程请求、父进程终止)

临界区

同步机制

同步准则
空闲让进、忙则等待、让权等待、有限等待

信号量 wait signal

整型信号量未完全遵守准则,不满足“让权等待”准则

互斥信号量 mutex,初值为1 
 

生产者消费者问题

锁 lock

管程
组成 名称、共享数据结构说明、过程、设置初始值
条件变量 condition 

AND信号量
信号量集

进程通信
低级工具 
高级工具 共享存储、消息传递、管道通信


引入线程
在操作系统中引入线程,是为了减少程序在并发执行时锁付出的时空开销,使OS具有更好的并发性,提高CPU的利用率。
进程是分配资源的基本单位,线程是系统调度的基本单位。

线程属性
轻型实体、独立调度和分派的基本单位、可并发执行、共享进程资源
一个进程的多个线程也可以并发执行



多线程OS中实现进程之间的同步与通信,提供的同步机制
互斥锁、读写锁、条件变量、信号

用户级线程
内核支持线程

高级调度、低级调度、中级调度
作业
作业 作业步 作业流

调度算法

作业调度与进程调度

死锁
产生原因与必要条件
原因 竞争资源和进程间推进顺序非法
必要条件 互斥条件、请求和保持、不可剥夺、循环等待

解决死锁方法
预防、避免、检测和解除
预防最容易实现
避免使资源利用率最高

银行家算法





\chapter{内存管理}
buddy

对换

分页存储管理
分段存储管理

快表
地址转换

虚拟存储器的特征
多次性、对换性、虚拟性,
虚拟性最本质

页表
页号、物理块号、状态位P、访问字段A、修改位M、外存地址

缺页中断
缺页次数

CLOCK
改进型CLOCK

中断处理过程

共享

逻辑地址到物理地址的转换
地址变换机构
快表


碎片

页
框
等术语的指代意义



\section{虚拟内存管理}
请求分页机制

\chapter{文件管理}

文件系统的结构及其实现,磁盘相关

文件系统
文件控制块、物理分配方法、索引结构、
磁盘
磁盘特性和结构、磁盘调度算法、磁盘相关性能

文件的逻辑结构
目录结构的实现
系统调用
文件的物理结构
存储空间的管理
文件共享保护

\section{文件系统}

\subsection{文件的逻辑结构}
文件的逻辑结构
	无结构文件(流式文件) 
	有结构文件(记录式文件) 
		顺序文件
			顺序存储或者链式存储
			串结构 顺序结构 
		索引文件
		索引顺序文件

\subsection{目录结构}
FCB的有序集合称为文件目录
一个FCB就是一个文件目录项

索引结点
磁盘索引结点、内存索引结点

引入无环图目录的目的是实现文件共享
无环图 共享计数器 计数器为0时才真正删除该结点


索引结点(FCB的改进)
FCB中暂时不用的信息放入索引结点中

索引结点机制,提升了文件检索速度


% 重点内容 
\subsection{文件物理结构}
即文件分配方式

文件的物理结构/文件分配方式
文件存储空间管理

索引分配
m级访问,m+1次读磁盘(顶级索引表未调入内存的情况下)

链接分配
索引分配

混合索引分配
Unix采用13个地址,0-9 直接地址,10-一级索引,11-二级索引,12-三级索引 

文件目录与目录文件

计算逻辑地址到物理地址的转换,以及查找次数。

文件共享
基于符号链的文件共享
软链接、硬链接


\subsection{文件存储空间管理}

空闲表法
空闲链表法
位示图法 字号,位号  常考 
成组链接法 UNIX超级块 








2020年4月14日凌晨观看王道2020视频教程

\include{cs/IO管理}


奈氏准则 理想低通 
极限码元传输率 2W Baud
极限数据传输率 2W log2V b/s



\include{cs/数据链路层}
\include{cs/网络层}
\chapter*{传输层}

\chapter*{应用层}



\end{document}
